\documentclass[12pt]{amsart}
\usepackage[margin=1in]{geometry}
\usepackage{amssymb}
\usepackage{stmaryrd}
\usepackage{graphicx}
\usepackage{todonotes}
\usepackage{hyperref}
\usepackage{url}

\usepackage{tensor}

\newcommand{\hoj}[1]{\todo[inline,color=blue!20]{HOJ: #1}}
\newcommand{\cjc}[1]{\todo[inline,color=red!20]{CJC:  #1}}
\newcommand{\je}[1]{\todo[inline,color=green!20]{JE:  #1}}
\newcommand{\dm}[1]{\todo[inline,color=yellow!20]{DM:  #1}}

\newcommand{\pder}[2]{\ensuremath{\frac{\partial #1}{\partial #2}}}
\newcommand{\so}{\ensuremath{\mathfrak{so}}}
\newcommand{\R}{\ensuremath{\mathbb{R}}}

\newtheorem{thm}{Theorem}[section]
\newtheorem{prop}[thm]{Proposition}
\newtheorem{lem}[thm]{Lemma}
\newtheorem{cor}[thm]{Corollary}
\newtheorem{defn}[thm]{Definition}

\DeclareMathOperator{\SDiff}{SDiff}
\DeclareMathOperator{\Diff}{Diff}
\DeclareMathOperator{\Jet}{Jet}
\DeclareMathOperator{\SO}{SO}
\DeclareMathOperator{\SL}{SL}
\DeclareMathOperator{\tr}{tr}
\DeclareMathOperator{\iso}{iso}
\DeclareMathOperator{\kernel}{kernel}
\DeclareMathOperator{\bag}{bag}
\DeclareMathOperator{\lie}{\mathcal{L}}
\DeclareMathOperator{\ad}{ad}
%%%%%%%%%%%%%%%%%%%%%%%%%%%%
\makeatletter
\newenvironment{modcases}{%
  \matrix@check\modcases\env@modcases
}{%
  \endarray\right.%
}
\def\env@modcases{%
  \let\@ifnextchar\new@ifnextchar
  \left.
  \def\arraystretch{1.2}%
  \array{@{}l@{\quad}l@{}}%
}
\makeatother

%%%%%%%%%%%%%%%%%%%%%%%%%%%%

%\renewcommand{\baselinestretch}{2} % Use for double spacing or (1.5} etc.
\addtolength{\parskip}{0.33\baselineskip} % 3X Increased spacing between paragraphs 

% Math notation
\def\al{\alpha} 
\def\be{\beta} 
\def\ga{\gamma} 
\def\de{\delta} 
\def\ep{\varepsilon} 
\def\ze{\zeta} 
\def\et{\eta} 
\def\th{\theta} 
\def\io{\iota} 
\def\ka{\kappa} 
\def\la{\lambda} 
\def\rh{\rho} 
\def\si{\sigma} 
\def\ta{\tau} 
\def\ph{\varphi} 
\def\ch{\chi} 
\def\ps{\psi} 
\def\om{\omega} 
\def\Ga{\Gamma} 
\def\De{\Delta} 
\def\Th{\Theta} 
\def\La{\Lambda} 
\def\Si{\Sigma} 
\def\Ph{\Phi} 
\def\Ps{\Psi} 
\def\Om{\Omega}
\def\GL{\operatorname{GL}}
\def\S{\operatorname{S}}
\def\Diff{\operatorname{Diff}}
\def\tr{\operatorname{tr}}
\def\X{\mathcal{X}}
\def\P{\mathcal{P}}
 
\def\o{\circ} 
\def\i{^{-1}} 
\def\x{\times}
\def\p{\partial} 
%\def\X{{\mathfrak X}}
\def\g{{\mathfrak g}}  
\def\L{\mathcal{L}}
\def\F{\mathcal{F}}
\def\H{\mathcal{H}}
\def\R{{\mathbb R}}
\def\ad{\operatorname{ad}} 
\def\Ad{\operatorname{Ad}}
\def\exp{\operatorname{exp}}
\def\one{\mathbbm{1}}
\let\on=\operatorname
\let\wt=\widetilde
\let\ol=\overline
\newcommand{\ud}{\,\mathrm{d}}



\pagestyle{myheadings}





\makeatother

\usepackage[english]{babel}

\begin{document}
\thispagestyle{empty}
\section*{{\bf Commuting group actions on $Q$ imply mutually conserved momentum maps}}

Let $Q$ be some manifold and $T^*Q$ its cotangent bundle. Write $\Phi^1$ and $\Phi^2$ for two actions of groups $G^i$ on $Q$, that is, $\Phi^i_g: Q \to Q$. Write $T\Phi^i_g: TQ \to TQ$ for the tangent lift of the action. Then the cotangent lifted action by some element $g$ is given by $(T\Phi^i_{g^{-1}})^*$. Assume the actions on $Q$ commute. 

\noindent \emph{Claim:} The cotangent lift momentum map $J^1: T^*Q \to \mathfrak{g}^1$ is invariant under the cotangent lift of the action $\Phi^2$. The same holds true with $1$ and $2$ interchanged.

\noindent \emph{Proof:} Let $h \in G^2$. Then we have, by the standard formula for cotangent lift momentum maps,
\begin{align}
	\left<J^1( (T\Phi^2_{h^{-1}})^* \alpha_q), \xi \right> = \left< (T\Phi^2_{h^{-1}})^* \alpha_q, \xi_Q( \Phi^2_h(q)\right>,
\end{align}
for any $\xi \in \mathfrak{g}^1$ and 
where the infinitesimal action that appears is the infinitesimal version of $\Phi^1$. The right hand side is further manipulated, as follows
\begin{align}
	&= \left<\alpha_q, T\Phi^2_{h^{-1}} \frac{d}{d\varepsilon}\Phi^1_{e^{\varepsilon \xi}}(\Phi_h^2 (q)) \right> =  \left<\alpha_q,  \frac{d}{d\varepsilon} \Phi^2_{h^{-1}} (\Phi^1_{e^{\varepsilon \xi}} (\Phi_h^2 (q)) )\right> \\
	&= \left<\alpha_q, \frac{d}{d\varepsilon} \Phi^1_{e^{\varepsilon \xi}}(q) \right> = \left<\alpha_q, \xi_Q(q) \right> = \left<J^1(\alpha_q), \xi\right>,
\end{align}
where in the first equality on the second line we used commutativity of the group actions. Since $\xi$ was an arbitrary element of $\mathfrak{g}^1$ we conclude that
\begin{align}
	J^1( (T\Phi^2_{h^{-1}})^* \alpha_q) = J^1(\alpha_q),
\end{align}
as promised.













\end{document}
 \\