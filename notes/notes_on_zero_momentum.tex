\documentclass[12pt]{amsart}
\usepackage{amsmath,amssymb}
\usepackage{geometry} % see geometry.pdf on how to lay out the page. There's lots.
\geometry{a4paper} % or letter or a5paper or ... etc
% \geometry{landscape} % rotated page geometry

%  POSSIBLY USEFULE PACKAGES
%\usepackage{graphicx}
%\usepackage{tensor}
%\usepackage{todonotes}

%  NEW COMMANDS
\newcommand{\pder}[2]{\ensuremath{\frac{ \partial #1}{\partial #2}}}
\newcommand{\ppder}[3]{\ensuremath{\frac{\partial^2 #1}{\partial
      #2 \partial #3} } }

%  NEW THEOREM ENVIRONMENTS
\newtheorem{thm}{Theorem}[section]
\newtheorem{prop}[thm]{Proposition}
\newtheorem{cor}[thm]{Corollary}
\newtheorem{lem}[thm]{Lemma}
\newtheorem{defn}[thm]{Definition}


%  MATH OPERATORS
\DeclareMathOperator{\SDiff}{SDiff}
\DeclareMathOperator{\iso}{iso}
\DeclareMathOperator{\GL}{GL}
\DeclareMathOperator{\SO}{SO}
\DeclareMathOperator{\ad}{ad}
\DeclareMathOperator{\Ad}{Ad}

%  TITLE, AUTHOR, DATE
\title{Some notes on $J^{-1}(0)$ and $T^*Q$}
\author{Henry O. Jacobs}
\date{\today}


\begin{document}

\maketitle

Let $G := \SDiff(\mathbb{R}^n)$, $H := \iso(z) < G$, $J$ the cotangent
lift momentum map of the right action of $H$ on $G$ and
$\pi\colon G \to Q := G/H$. We can write $J^{-1}(0)$ as a set
\begin{align*}
	J^{-1}(0) = \{ p_g \in T^*G \mid \langle p_g , g \cdot w \rangle = 0 \quad \forall w \in \mathfrak{h} \}
\end{align*}
Then $J^{-1}(0) / H$ is
\begin{align*}
	J^{-1}(0) / H = \{ [p_g] \in T^*G/H \mid  \langle p_g , g \cdot w \rangle = 0 \quad \forall w \in \mathfrak{h} \},
\end{align*}
where brackets denote the equivalence classes in $T^*G/H$ and $TG/H$.

Similarly, we see that $i(T^*Q)$ is the set
\begin{align*}
	i(T^*Q) = \{ [p_g] \in T^*G/H \mid \exists p_q \in T^*Q\colon \forall v_g \in T_g G\colon \langle [p_g] , [v_g] \rangle = \langle p_q , T_g\pi(v_g) \rangle \}
\end{align*}
Note that the kernel of $T_g\pi$ is the set $\{ g \cdot w \mid w \in \mathfrak{h}\}$.
Moreover, $T_g \pi$ is linearly surjective.
This tells us that the constraint
\begin{align*}
	\langle [p_g] , [v_g] \rangle = \langle p_q , T_g\pi(v_g) \rangle \text{ for some } p_q \in T^*Q
\end{align*}
is equivalent to the constraint
\begin{align*}
	\langle [p_g], [g \cdot w] \rangle = 0 \quad \forall w \in \mathfrak{h}
\end{align*}
This the sets $J^{-1}(0) / H $ and $i(T^*Q)$ are identical.


\bibliographystyle{amsalpha}
\bibliography{hoj.bib}
\end{document}
