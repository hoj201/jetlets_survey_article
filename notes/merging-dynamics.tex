\documentclass[a4paper,11pt]{article}

\usepackage[margin=2.5cm]{geometry}
\usepackage{amsmath,amssymb}
\usepackage{tikz}
\usetikzlibrary{arrows}
\usetikzlibrary{matrix}
\usetikzlibrary{cd}

\title{Dynamical aspects of jetlet mergers}
\author{Jaap Eldering}

\renewcommand{\d}{\ensuremath{\,\textrm{d}}}
\newcommand{\R}{\ensuremath{\mathbb{R}}}

\begin{document}

\maketitle

We redo the symplectic reduction of the two $0$-jetlets. The
Hamiltonian is
\begin{equation}\label{eq:H-two-0jetlets}
  H = \frac{1}{2} \sum_{a,b=1}^2 p_{ai} K^{ij}(q_a - q_b) p_{bj}
\end{equation}
on $T^* \R^{2n}$ with the canonical Poisson brackets. We switch to a
\emph{slightly different} center of mass frame by choosing new
coordinates
\begin{equation}\label{eq:CoM-coords}
  \bar{q} = \frac{1}{2}(q_1 + q_2), \qquad \tilde{q} = q_2 - q_1
\end{equation}
with canonically associated momenta
\begin{equation*}
  \bar{p} = p_1 + p_2, \qquad \tilde{p} = \frac{1}{2}(p_2 - p_1).
\end{equation*}
The Hamiltonian in these cooordinates becomes
\begin{equation*}
  H = \frac{1}{4} \bar{p}_i \big( K^{ij}(0) + K^{ij}(\tilde{q}) \big)   \bar{p}_j
              + \tilde{p}_i \big( K^{ij}(0) - K^{ij}(\tilde{q}) \big) \tilde{p}_j.
\end{equation*}
Under the assumption that $\bar{p} = 0$ we reduce further, introducing
polar coordinates
\begin{equation}\label{eq:polar-coords}
  \tilde{q} = r \big(\cos(\varphi),\sin(\varphi)\big)
\end{equation}
with canonically associated momenta
\begin{equation*}
  \tilde{p} =
  \begin{pmatrix}
    \cos(\varphi) & -\frac{\sin(\varphi)}{r} \\
    \sin(\varphi) &  \frac{\cos(\varphi)}{r}
  \end{pmatrix} \cdot
  \begin{pmatrix}
    p_r \\
    p_\varphi
  \end{pmatrix}
  = R_\varphi \cdot
  \begin{pmatrix}
    p_r \\
    \frac{p_\varphi}{r}
  \end{pmatrix},
\end{equation*}
where $R_\varphi$ is a rotation matrix. In these coordinates the
Hamiltonian is given by
\begin{equation*}
  \begin{aligned}
  H &= \tilde{p}_i \big[ K^{ij}(0) - K^{ij}(\tilde{q}) \big] \tilde{p}_j
     = \begin{pmatrix} p_r \\ \frac{p_\varphi}{r} \end{pmatrix}^T R_\varphi^T
       \big[ K(0) - K(R_\varphi\cdot(r,0)) \big] R_\varphi
       \begin{pmatrix} p_r \\ \frac{p_\varphi}{r} \end{pmatrix}\\
    &= \begin{pmatrix} p_r \\ \frac{p_\varphi}{r} \end{pmatrix}^T
       \big[ K(0) - K(r,0) \big] \begin{pmatrix} p_r \\ \frac{p_\varphi}{r} \end{pmatrix},
  \end{aligned}
\end{equation*}
where in the last step we used that $K$ as a tensor is invariant under
rotations. Since $\varphi$ is a cyclic variable, we find that its
associated momentum
\begin{equation*}
  p_\varphi = -r \sin(\varphi) \tilde{p}_1 + r \cos(\varphi) \tilde{p}_2
            = \tilde{q} \wedge \tilde{p}
\end{equation*}
is conserved.

Let us now choose the smooth kernel $K = K_{\infty,1}$ given
(up to a scaling factor) by
%
{\newcommand{\e}{e^{-\rho}}
\begin{equation}\label{eq:smooth-kernel}
  K^{ij}(x) = \Big(\e - \frac{1}{2\rho}\big(1 - \e\big)\Big)\delta^{ij}
    + \Big(\frac{1}{\rho}\big(1 - \e\big) - \e\Big)\frac{x^i x^j}{\|x\|^2},
\end{equation}%
}%
where $\rho := \frac{\|x\|^2}{2}$. Note that
$K^{ij}(0) = \frac{1}{2} \delta^{ij}$ and $\partial_k K^{ij}(0) = 0$.
Using rotational symmetry we set $\varphi = 0$, and obtain
\begin{equation}\label{eq:H-rot-reduced}
  H = \frac{p_r^2      }{2}    \Big(1 - \frac{1}{\rho}\big(1-e^{-\rho}\big)\Big)
     +\frac{p_\varphi^2}{4\rho}\Big(1 - 2e^{-\rho} + \frac{1}{\rho}\big(1-e^{-\rho}\big)\Big).
\end{equation}
To analyse the asymptotic behaviour of the system in the limit that
$r \to 0$, we make an expansion of $H$ around $\rho = 0$:
\begin{equation}
  H = \frac{p_r^2      }{2}\big(\frac{1}{2}\rho - \frac{1}{6}\rho^2\big)
     +\frac{p_\varphi^2}{2}\big(\frac{3}{4} - \frac{5}{12}\rho + \frac{7}{48}\rho^2\big)
     +\mathcal{O}(\rho^3).
\end{equation}
Since $H$ and $p_\varphi^2$ are preserved, we can solve for $p_r$ in
terms of $r$, and we find
\begin{align*}
  2\rho\,p_r^2 &= (8H - 3 p_\varphi^2) \big(1 + \mathcal{O}(\rho)\big), \\
\Longleftrightarrow \qquad
      r\,p_r   &= \sqrt{8H - 3 p_\varphi^2} + \mathcal{O}(r).
\end{align*}
We write $\xi := \sqrt{8H - 3 p_\varphi^2}$ and thus obtain
asymptotically $p_r = \frac{\xi}{r}$. Further, we have dynamics
\begin{equation}
  \dot{r} = \frac{\partial H}{\partial p_r} = \frac{1}{4}p_r\,^2 + \mathcal{O}(r^4), \qquad
  \dot{\varphi} = \frac{\partial H}{\partial p_\varphi}
                 = \frac{3}{4}p_\varphi + \mathcal{O}(r^2)
\end{equation}
and reconstruct
\begin{alignat*}{2}
  \bar{q}   &= 0,& \qquad
  \tilde{q} &= r R_\varphi \cdot\begin{pmatrix} 1 \\ 0 \end{pmatrix},\\
  \bar{p}   &= 0,& \qquad
  \tilde{p} &= R_\varphi \cdot
               \begin{pmatrix} p_r \\ \frac{p_\varphi}{r} \end{pmatrix}
             = \frac{1}{r} R_\varphi \cdot
               \begin{pmatrix} \xi \\ p_\varphi \end{pmatrix} + \mathcal{O}(1).
\end{alignat*}
Now we consider the image under $J_L$ of the asymptotic solution
curve:
\begin{align*}
  J_L(q_1,p_1,q_2,p_2)
  &= p_1 \otimes \delta_{q_1} + p_2 \otimes \delta_{q_2} \\
  &= \bar{p} \otimes \delta_{\bar{q}}
    +\tilde{p} \otimes \big(\tilde{q} \cdot D\delta_{\bar{q}}\big)
    +\mathcal{O}\Big(\big(|\bar{p}|+|\tilde{p}|\big)|\tilde{q}|^2\Big) \\
  &= R_\varphi \cdot \begin{pmatrix} \xi \\ p_\varphi \end{pmatrix} \otimes
     R_\varphi \cdot \begin{pmatrix} 1 \\ 0 \end{pmatrix} D\delta_{\bar{q}}
    +\mathcal{O}(r)
\end{align*}
with $\varphi(t) = p_\varphi\,t$. The factor in front of $D\delta$
should correspond to $\mu(t)$ for a $1$-jetlet with position
$q^{(0)} = 0$ and momentum $p^{(0)} = 0$. For such a setup we have
equations of motion
\begin{equation}
  \dot{\mu}_i^j
  = \partial_k u^j(0) \mu_i^k + \mu_k^j \partial_i u^k(0)
  = -\partial_{km} K^{jl}(0) \mu_l^m \mu_i^k
    -\partial_{im} K^{kl}(0) \mu_l^m \mu_k^j.
\end{equation}


\clearpage

A more abstract way to view these particle mergers, is to think of how
our hierarchy of reduced spaces $T^*Q_N^{(k)}$ embeds into
$\mathfrak{X}_{\rm div}(\R^n)^*$ under the momentum map $J_L$.
Consider a merging pair of $0$-jetlets, described by a curve
$x_0(t) \in T^*Q_2^{(0)}$. As the particles approach each other,
$x_0(t)$ approaches the boundary of $T^*Q_2^{(0)}$ given by
\begin{equation}\label{eq:boundary}
  \partial T^*Q_2^{(0)}
  = \{ (q_1,p_1,q_2,p_2) \in T^* \R^n \times T^* \R^n \mid q_1 = q_2 \}.
\end{equation}
On the other hand, if we consider the image curve
$y_0(t) = J_L(x_0(t)) \in \mathfrak{X}_{\rm div}(\R^n)^*$, then this
can be compared to images under $J_L$ of other space $T^*Q_N^{(k)}$.
Since $y_0(t)$ consists of two covector-valued delta distributions at
$q_1,q_2$, in the limit as their distance goes to zero, this can be
approximated by a momentum valued distribution of a delta and its
derivative, i.e.~an element in $J_L(T^*Q_1^{(1)})$.

\begin{equation}
  \begin{tikzcd}
    T^*Q_N^{(k)} \ar[to=1-2,hook, "J_L"] & \mathfrak{X}_{\rm div}(\R^n)^* \\
    T^*Q_1^{(1)} \ar[to=1-2,hook,start anchor=north east,end anchor={[yshift=-0.5em]west}]
                 \ar[u,dotted,no head] &  \\
    T^*Q_2^{(0)} \ar[to=1-2,hook,start anchor=north east,end anchor={[yshift=-1.0em]west}]
                 \ar[u,"\partial"] &
  \end{tikzcd}
\end{equation}



\end{document}

%%% Local Variables: 
%%% mode: latex
%%% TeX-master: t
%%% TeX-PDF-mode: t
%%% End: 
