\documentclass{article}
\usepackage{amsmath, amssymb}
\usepackage{amsthm}
\providecommand{\norm}[1]{\lVert#1\rVert}
\providecommand{\Norm}[1]{\big\lVert#1\big\rVert}
\setlength{\textwidth}{490pt}
\setlength{\textheight}{60\baselineskip}
%\addtolength{\textheight}{\topskip}
\setlength{\oddsidemargin}{-0.9cm}
\setlength{\topmargin}{-1.5cm}

\usepackage{color,bm}
\pagestyle{empty}


\def\Q{\mathbb{Q} }
\def\R{\mathbb{R} }
\def\Z{\mathbb{Z} }
\def\nbd{neighborhood }
\def\nbds{neighborhoods }
\def\R{\mathbb{R} }
\def\iff{if and only if }

\DeclareMathOperator{\SDiff}{SDiff}
\DeclareMathOperator{\Jet}{Jet}
\DeclareMathOperator{\iso}{iso}

\usepackage{todonotes}

\begin{document}

\section{Clarifications to comments on the revised paper}

\subsection{Proof of dual pair in Proposition 4.4}


\subsection{Reviewer 2 comments on Section 4.5}

Reviewer 2 mentions a number of unclarities in Section 4.5 ``Kelvin's
circulation theorem''. Below the comments and our responses.

\begin{itemize}
\item p.58 line 59--60 and p.57 line 1. It states ``the statement
  $\pi(p_\varphi)) = i(p_q)$ for some $p_\varphi \in T^*\SDiff(\R^n)$
  and \ldots $\Jet_z^{(k)}(w \circ \varphi)\rangle$.''.
  The equivalence is not obvious to me. Please clarify.

\item p.58 line 43 and line 53. it's not obvious the definition of
  $T^*\SDiff(\R^n)/\iso^{(k)}(z)$.
  Please describe the definition. Or should
  ``$T^*\SDiff(\R^n)/\iso^{(k)}(z)$''
  be ``$T^*\SDiff(\R^n)/\mathfrak{iso}^{(k)}(z)$''?

  The quotient is by the cotangent-lifted (right) action of
  $\iso^{(k)}(z)$. We have clarified this in the text.

\item p.58 line 49. In this context, it's not obvious to me whether
  this $\mathfrak{iso}^*(z)$ is the dual of $\iso^{(k)}(z)$ or
  $\iso(z)$. Please describe the definition.

  It is the dual of $\iso^{(k)}(z)$; this was a sloppy
  notation oversight. We clarified this by explicitly writing
  $\mathfrak{iso}^{(k)}(z)^*$.

\item p.58 line 43. In this context, it's not obvious that ``we can
  consider the element $i(q, p) \in T^*\SDiff(\R^n)$ \ldots (2.2.4)].''.
  Please describe in detail. Or should ``$i(q,p) \in T^*\SDiff(\R^n)$''
  be ``$i(q,p) \in T^*\SDiff(\R^n) / \iso^{(k)}(z)$''?

  Indeed, this was a typo and we have corrected it to
  ``$i(q,p) \in T^*\SDiff(\R^n) / \iso^{(k)}(z)$''.
\end{itemize}

\section{Typos}
Reviewer 2 found the following (extra) typos. We could not exactly
reproduce the line numbers given by the reviewer; in parentheses is
specified the text position we assumed was meant.
\begin{itemize}
\item p.56 line 17 (line 32--33). ``$\delta_{\gamma_n}^{[b_n]}$'' should be ``$\delta_{\gamma_n}^{[j_n]}$''.
\item p.69 line 19 (line 40). ``transitive'' should be ``infinitesimally transitive''.
\item p.69 line 27 and line 34. I think that ``$\rho_1^*$'' should be
  ``$\rho_1$'', because of $\rho_1^*: G_1 \times T^*Q \to T^*Q$.

  We assume the reviewer refers to the use of ``$\rho_1^*$'' in
  Lemma~A.10 and its proof. We think this is technically correct, but
  confusing and inconsistent with previously established notation that
  ``$\rho_1$'' is the cotangent-lifted action, so we changed all
  instances of ``$\rho_1^*$'' into ``$\rho_1$''.
\item p.69 line 34 (line 56). ``$q' = g \cdot q$'' should be ``$q' = g^{-1} \cdot q$''.
\end{itemize}

\noindent
We agreed with these findings and changed them in the revised manuscript.


\end{document}
