\documentclass{article}
\usepackage{amsmath, amssymb}
\usepackage{amsthm}
\providecommand{\norm}[1]{\lVert#1\rVert}
\providecommand{\Norm}[1]{\big\lVert#1\big\rVert}
\setlength{\textwidth}{490pt}
\setlength{\textheight}{60\baselineskip}
%\addtolength{\textheight}{\topskip}
\setlength{\oddsidemargin}{-0.9cm}
\setlength{\topmargin}{-1.5cm}

\usepackage{color,bm}
\pagestyle{empty}


\def\Q{\mathbb{Q} }
\def\R{\mathbb{R} }
\def\Z{\mathbb{Z} }
\def\nbd{neighborhood }
\def\nbds{neighborhoods }
\def\R{\mathbb{R} }
\def\iff{if and only if }

\DeclareMathOperator{\SDiff}{SDiff}
\DeclareMathOperator{\Jet}{Jet}
\DeclareMathOperator{\iso}{iso}
\DeclareMathOperator{\kernel}{kernel}

\usepackage{todonotes}

\begin{document}

\section{Clarifications to comments on the revised paper}

\subsection{Proof of dual pair in Proposition 4.4}

Reviewer 3 pointed out that there was still a problem with our proof
that $J_L^{(k)}$ and $J_R^{(k)}$ form a dual pair.

We agree with the reviewer's observation and after closer inspection
we are convinced that $J_L^{(k)}$ and $J_R^{(k)}$ indeed do not form
dual pair for $k \ge 1$, although they do for $k = 0$. However, for
$k \ge 1$ they still form a weak dual pair, as defined and extensively
studied in the paper ``Dual pairs in fluid dynamics'' (2012) by
Gay-Balmaz and Vizman. This weakened definition only requires
\begin{equation*}
  \kernel(J_L^{(k)})^\omega \subset \kernel(J_R^{(k)}), \qquad
  \kernel(J_R^{(k)})^\omega \subset \kernel(J_L^{(k)}),
\end{equation*}
instead of strict equalities and is sufficient to prove Theorem~A.9
without modification (Theorem A.8 in the original version).

We have replaced `dual pair' by `weak dual pair' in appropriate places
with some remarks and modified our proofs. This actually simplified
the proofs, since we need only prove that the left and right actions
of $\SDiff(\R^n)$ and $\iso(z)$ commute to use Corollary~2.6 from the
Gay-Balmaz and Vizman paper.

\subsection{Reviewer 2 comments on Section 4.5}

Reviewer 2 mentions a number of unclarities in Section 4.5 ``Kelvin's
circulation theorem''. Below the comments and our responses.

\begin{itemize}
\item p.58 line 59--60 and p.57 line 1. It states ``the statement
  $\pi(p_\varphi)) = i(p_q)$ for some $p_\varphi \in T^*\SDiff(\R^n)$
  and \ldots $\Jet_z^{(k)}(w \circ \varphi)\rangle$.''.
  The equivalence is not obvious to me. Please clarify.
  
We've added greater detail to this section on the whole.
In fact, we suspect this request of the reviewer is indicative of an over-arching lack of clarity for the entire section on Kelvin's circulation theorem.
We've addressed this deficiency in the paper by stating the maps involved more explicitly.

In direct response to the reviewer's question though, the following should help clarify things.
Recall $\pi: T^*\SDiff(\R^n) \to T^*\SDiff(\R^n) / \iso^{(k)}(z)$ is the quotient map, so $\pi$ is surjective by construction.
Also recall $i: T^*Q \to T^*\SDiff(\R^n) / \iso^{(k)}(z)$.
Therefore, for any $(q,p) \in T^*Q$ there must exist a $p_\varphi \in T^* \SDiff(\R^n)$ such that $\pi( \pi_\varphi) = i(q,p)$.

\item p.58 line 43 and line 53. it's not obvious the definition of
  $T^*\SDiff(\R^n)/\iso^{(k)}(z)$.
  Please describe the definition. Or should
  ``$T^*\SDiff(\R^n)/\iso^{(k)}(z)$''
  be ``$T^*\SDiff(\R^n)/\mathfrak{iso}^{(k)}(z)$''?

  $T^*\SDiff(\R^n) / \iso^{(k)}(z)$ refers to the quotient of $T^*\SDiff(\R^n)$
  by the cotangent-lifted (right) action of
  $\iso^{(k)}(z)$. We have clarified this in the text.

\item p.58 line 49. In this context, it's not obvious to me whether
  this $\mathfrak{iso}^*(z)$ is the dual of $\iso^{(k)}(z)$ or
  $\iso(z)$. Please describe the definition.

  It is the dual of $\mathfrak{iso}(z)$.
  Here $J_{\rm conv}$ refers to the action of $\iso(z)$ on $T^* \SDiff(\R^n)$.
  Perhaps the reviewer brought up this question because we are doing a reduction by the normal subgroup $\iso^{(k)}(z) \trianglelefteq \iso(z)$, \emph{not} a reduction by $\iso(z)$.
  More generally, the reviewer has brought to light that these deliberate choices were too subtle, and could be confused with typos.
  We've decided to unfold the details that were largely implicit and perhaps hidden in a terse commutative diagram.
  Even the statement of the commutative diagram was not obvious.
  What we found helped was to add two additional maps, so that the diagram was now broken into commutative triangles, which have a more pedestrian interpretation.
  Moreover, the diagram is now explained and its commutativity is verified in greater detail in the text.

\item p.58 line 43. In this context, it's not obvious that ``we can
  consider the element $i(q, p) \in T^*\SDiff(\R^n)$ \ldots (2.2.4)].''.
  Please describe in detail. Or should ``$i(q,p) \in T^*\SDiff(\R^n)$''
  be ``$i(q,p) \in T^*\SDiff(\R^n) / \iso^{(k)}(z)$''?

  Indeed, this was a typo and we have corrected it to
  ``$i(q,p) \in T^*\SDiff(\R^n) / \iso^{(k)}(z)$''.
\end{itemize}

\clearpage
\section{Typos}
Reviewer 2 found the following (extra) typos. We could not exactly
reproduce the line numbers given by the reviewer; in parentheses is
specified the text position we assumed was meant.
\begin{itemize}
\item p.56 line 17 (line 32--33). ``$\delta_{\gamma_n}^{[b_n]}$'' should be ``$\delta_{\gamma_n}^{[j_n]}$''.
\item p.69 line 19 (line 40). ``transitive'' should be ``infinitesimally transitive''.
\item p.69 line 27 and line 34. I think that ``$\rho_1^*$'' should be
  ``$\rho_1$'', because of $\rho_1^*: G_1 \times T^*Q \to T^*Q$.

  We assume the reviewer refers to the use of ``$\rho_1^*$'' in
  Lemma~A.10 and its proof. We think this is technically correct, but
  confusing and inconsistent with previously established notation that
  ``$\rho_1$'' is the cotangent-lifted action.
\item p.69 line 34 (line 56). ``$q' = g \cdot q$'' should be ``$q' = g^{-1} \cdot q$''.
\end{itemize}

\noindent
We agreed with these findings. The first we changed in the revised
manuscript, while the others are not relevant anymore since we removed
Theorem~A.9 and Lemma~A.10 following the issue pointed out by reviewer 3.

\end{document}
