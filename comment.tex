\documentclass{article}
\usepackage{amsmath, amssymb}
\usepackage{amsthm}
\providecommand{\norm}[1]{\lVert#1\rVert}
\providecommand{\Norm}[1]{\big\lVert#1\big\rVert}
\setlength{\textwidth}{490pt}
\setlength{\textheight}{60\baselineskip}
%\addtolength{\textheight}{\topskip}
\setlength{\oddsidemargin}{-0.9cm}
\setlength{\topmargin}{-1.5cm}

\usepackage{color,bm}
\pagestyle{empty}


\def\Q{\mathbb{Q} } 
\def\R{\mathbb{R} } 
\def\Z{\mathbb{Z} } 
\def\nbd{neighborhood } 
\def\nbds{neighborhoods } 
\def\R{\mathbb{R} } 
\def\iff{if and only if } 

\usepackage{todonotes}

\begin{document}






\section{Clarifications}

\subsection{Transitivity}
We agreed with the reviewer's claim that the transitivity arguments being used were a problem.  Reviewer 2 specifically pointed to p.29 line 20, p.13 line 33-34, and
p.16 line 56-57.
The reason reviewer 2 was having trouble here was identified by reviewer 3.  He/she found the claims to be false.  Fortunately, this was not critical to the propositions and lemmas in which these claims were used.  A corrected version of these proofs now appears in the paper.  We found that all that was required for the propositions to hold was that certain group actions be transitive on level sets of momentum map, which follows quickly using the added Lemma~A.10. We did not require transitivity on T*Q as a whole (which we did not have, and should never have utilized to begin with). Thank you reviewers for catching this glitch!

\subsection{Differentiability}
The second reviewer found inconsistencies with respect to the differentiability requirements on the Hamiltonians in the paper.
For reasons that our lost on me as well, many of these Hamiltonian's were said to be in $C^{2}$ when all that was required was that they be in $C^{1}$.
This is merely a necessary conditions in order for the weak form of Hamilton's equations to be well defined.

\subsection{p.17 line 52}
 Here we used the phrase ``By the general principles discussed earlier".
 Reviewer 2 requested we be less vague here.  We've replaced this hand-waving phrase with a reference to the main theorem used in the text (Theorem A.8).  We've also written the correspondence between momentum maps in this section and those in the statement of Theorem A.8.

\subsection{p.19 line 4}
He we use the phrase ``recall that we have shown in the previous section...". 
Reviewer 2 objected to our hand-waving here.
Upon further inspection, the use of the equation which followed this section was not needed, and so this sentence was removed as well.

This section of the text was related to verifying the commutativity of a diagram.  We have replaced the original explanation with a more precise and succint one.

\subsection{Jet merger calculations}
Reviewer 2 objected to ``$\dot{r} = \cdots = 1/4 p_r^2 + \cdots$'' in
equation (20). Indeed, this was a typo and we corrected it to
\begin{equation*}
  \dot{r} = \frac{\partial H}{\partial p_r} = 1/4 p_r\,r^2 + \mathcal{O}(r^4).
\end{equation*}

On page 22 line 42, reviewer 2 found the derivation of the equation
from (17) not clear. Indeed some intermediate steps were missing,
mainly because this is a tedious calculation which we performed with
Mathematica. We have noted this and inserted an intermediate explicit
expression of the tensor $\partial_{ij} K^{kl}(0)$.

\section{Typos}
Reviewer 2 found the following typos:
\begin{itemize}
 \item p.12 line 57. I think that ``$\delta_{q_a}$" should be ``$\delta_{q_a^{(0)}}$". 
\item p.18 line 4. ``$\gamma(s)$" should be ``$\gamma_0(s)$". 
\item p.18 line 46,47. ``$(p, q)$'' should be ``$(q, p)$". 
\item p.26 line 7. the third ``$G$" should be ``$F$".  
\item p.28 line 17. ``$\{f, g \}_2$''' should be ``$\{f, g \}_2 \circ \psi$''. 
\item p.29 line 40. ``$\rho_1(\xi_2)$'' should be ``$\rho_2(\xi_2)$''. 
\item p.29 line 46. The sign of omega should be opposite and ``$(x, \xi_2)$'' 
should be ``$(\xi_2, x)$''. 
\\
\item p.29 line 48,51,56. 
``$\mathfrak{X}(\mathbb{R}^n)$'' should be 
``$\mathfrak{X}(\mathbb{R}^n)^*$''. 
\end{itemize}

We agreed with these findings and changed them in the revised manuscript.

The following we think are actually correct:
\begin{itemize}
\item p.22 line 46. I think that a sign is opposite: ``$+ 1/2$" should be ``$-1/2$".\\
  Reviewer 2 was correct that there should be a minus sign when
  deriving this from previous expressions. However, on closer
  inspection it turned out that there were two other minus sign
  mistakes: one in the expression for $\xi$ on line 31 (which cancels
  the mistake noted by the reviewer) and in the tensor decomposition
  of $\partial_{ij} K^{kl}(0)$ on line 42, the term
  $\omega \otimes \omega$ missed a minus sign. Together with a missing
  transposition of $\xi$ in formula (25) these end up yielding the
  same final result, as is also expected from numerical evidence.
\item p.24 line 18,19. ``$6$" should be ``$3$".\\
  The `$6$' is correct given the choice of coordinate
  transformations in (15) and (16) and the corresponding formulas for
  momenta below them. In particular note that $\tilde{q} = q_2 - q_1$
  without a factor $1/2$ to make the transformation symplectic.
\end{itemize}



\end{document}


